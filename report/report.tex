\documentclass[12pt,a4paper]{article}
\usepackage[utf8]{inputenc}
\usepackage[english]{babel}
\usepackage{amsmath}
\usepackage{amsfonts}
\usepackage{amssymb}
\usepackage{graphicx}
\author{José Antonio García Hernández}
\title{$\lambda \phi^4$ lattice field theory with a U(1) gauge field}
\begin{document}
\maketitle

\section{Lagrangian}
We begin by writing the Lagrangian in Euclidian spacetime with a complex scalar field $\phi(x)\in\mathbb{C}$
\begin{equation}
    \mathcal{L} = \frac{1}{2}\left(\partial_{\mu}\phi(x)\right)^* \left(\partial_{\mu}\phi(x)\right)+ \frac{1}{2}m^2\phi(x)^*\phi(x) + \frac{1}{4}\lambda\left(\phi(x)^*\phi(x)\right)^2,
\end{equation}
where $\lambda>0$. This Lagrangian posses a U(1) global symmetry $\phi(x) \to \phi(x)' = \Omega\phi(x)$ with $\Omega \in \text{U(1)}$.

We can rewrite the Lagrangian as follows
\begin{equation}
    \mathcal{L} = -\frac{1}{2}\left(\phi(x)\partial_{\mu}\right)^*\partial_{\mu}\phi(x) + \frac{1}{2}m^2\phi(x)^*\phi(x) + \frac{1}{4}\lambda\left(\phi(x)^*\phi(x)\right)^2 ,
\end{equation}
where we have dropped the total derivative term since it will integrate to zero in the action.

Now we add a gauge field $A_{\mu}(x)\in\mathbb{R}$, that implements the U(1) symmetry locally, and replace the partial derivatives with the covariant derivative defined as
\begin{equation}
    D_{\mu} \equiv \partial_{\mu} + iA_{\mu}(x).
\end{equation}

The Lagrangian now reads
\begin{equation}
    \mathcal{L} = -\frac{1}{2}\left(\phi(x) D_{\mu}\right)^*D_{\mu}\phi(x) + \frac{1}{2}m^2\phi(x)^*\phi(x) + \frac{1}{4}\lambda\left(\phi(x)^*\phi(x)\right)^2 + \frac{1}{4}F_{\mu\nu}F_{\mu\nu}.
\end{equation}

The discrete version of the Lagrangian takes the form 
\begin{equation}
    \mathcal{L} = -\frac{1}{2}\left(\phi_x D_{\mu}^-\right)^*D^+_{\mu}\phi_x + \frac{1}{2}m^2\phi_x^*\phi_x + \frac{1}{4}\lambda\left(\phi_x^*\phi_x\right)^2 + \beta\sum_{\mu<\nu}\text{Re} \left[1 - U_{x,\mu\nu} \right].
\end{equation}
In order to preserve gauge invariance we have used two discrete versions of the covariant derivative
\begin{eqnarray}
    D^+_{\mu}\phi_x = U_{x,\mu}\phi_{x+\hat{\mu}} - \phi_x ,\\
    D^-_{\mu}\phi_x = \phi_{x} - U_{x-\hat{\mu},\mu} \phi_{x-\hat{\mu}}. 
\end{eqnarray}

The final form of the Lagragian reads
\begin{equation}
    \mathcal{L} = \frac{1}{2}\sum_{\mu}\phi_x^*\left( 2\phi_x - U^*_{x-\hat{\mu},\mu}\phi_{x-\hat{\mu}} - U_{x,\mu}\phi_{x+\hat{\mu}}\right) + \frac{1}{2}m^2\phi_x^*\phi_x + \frac{1}{4}\lambda\left(\phi_x^*\phi_x\right)^2 + \beta\sum_{\mu<\nu}\text{Re} \left[1 - U_{x,\mu\nu} \right].
\end{equation}

And the action reads
\begin{eqnarray}
    S[\phi,U] & = &\sum_x \left\{ d|\phi_x|^2 + \frac{1}{2}m^2|\phi_x|^2 + \frac{1}{4}\lambda |\phi_x|^4 \right\} \nonumber\\
    & & - \sum_x\sum_{\mu}\text{Re}\left[\phi_x^*\left(U^*_{x-\hat{\mu},\mu}\phi_{x-\hat{\mu}} + U_{x,\mu}\phi_{x+\hat{\mu}}\right)\right]\nonumber \\     
     & & + \beta\sum_x\sum_{\mu<\nu}\text{Re} \left[1 - U_{x,\mu\nu} \right].
\end{eqnarray}

\subsection{Updating the fields}
In the Metropolis algorithm we need the difference in the action by a local update in a field.

An update of the scalar field $\phi_x \to \phi'_x$ leads to 
\begin{eqnarray}
    \Delta S & = & \left(d +\frac{m^2}{2} \right)\left( |\phi'_x|^2 - |\phi_x|^2\right) + \frac{\lambda}{4}\left( |\phi'_x|^4 - |\phi_x|^4\right) \nonumber\\
    & &- 2\text{Re}\left[\sum_{\mu}\left(\phi'_x -\phi_x \right)^*\left(U^*_{x-\hat{\mu},\mu}\phi_{x-\hat{\mu}} + U_{x,\mu}\phi_{x+\hat{\mu}}\right) \right].
\end{eqnarray}

An update of the gauge field $U_{x,\mu} \to U'_{x,\mu}$ leads to
\begin{equation}
    \Delta S = -\text{Re} \left[ (U'_{x,\mu} - U_{x,\mu})\left(\beta\Sigma^*_{x,\mu}+2\phi^*_x \phi_{x+\hat{\mu}}\right)\right]
\end{equation}

\subsection{Case $\lambda = 0$}
We define the operator $\square$
\begin{equation}
	\square_{x,y} \equiv (2+m)\delta_{x,y} - U^*_{x-\hat{\mu},\mu}\delta_{x-\hat{\mu},y} - U_{x,\mu}\delta_{x+\hat{\mu},y}  
\end{equation}

In matrix form,
\begin{equation}
\square = \begin{pmatrix}
	2 + m & -U_{(1,1),2}   & 0 & -U^*_{(1,4),2} & \cdots & 0  \\
	 -U^*_{1} & 2+m & -U_{2} & \cdots & 0 & 0\\
	 0 & -U^*_{2} & 2+m & -U_{3} & \cdots & 0 \\
	 \vdots \\
	-U_{L} & 0 & \cdots & 0& -U^*_{L-1} & 2+m & 
\end{pmatrix}
\end{equation}
We now write the action in matrix form
\begin{equation}
	S = \vec{\phi}^{\dagger}\square\vec{\phi} + \beta\sum_{\mu<\nu}\text{Re} \left[1 - U_{x,\mu\nu} \right]
\end{equation}


\end{document}

